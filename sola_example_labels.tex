This file contains the command names and the labels used in
the associated .tex file.    It is useful to have present file open
as the same time of the .tex file to copy/paste definitions.
When a new definition is added, it is important to keep this file
updated. 

   % Copy/paste of the definitions used in the .tex
\newcommand{\BibTeX}{\textsc{Bib}\TeX}
\newcommand{\etal}{et al.}

% Definitions for equations
\newcommand{\deriv}[2]{\frac{{\mathrm d} #1}{{\mathrm d} #2}}
\newcommand{\rmd}{ {\ \mathrm d} }
\renewcommand{\vec}[1]{ {\mathbf #1} }
\newcommand{\uvec}[1]{ \hat{\mathbf #1} }
\newcommand{\pder}[2]{ \f{\partial #1}{\partial #2} }
\newcommand{\grad}{ {\bf \nabla } }
\newcommand{\curl}{ {\bf \nabla} \times}
\newcommand{\vol}{ {\mathcal V} }
\newcommand{\bndry}{ {\mathcal S} }
\newcommand{\dv}{~{\mathrm d}^3 x}
\newcommand{\da}{~{\mathrm d}^2 x}
\newcommand{\dl}{~{\mathrm d} l}
\newcommand{\dt}{~{\mathrm d}t}
\newcommand{\intv}{\int_{\vol}^{}}
\newcommand{\inta}{\int_{\bndry}^{}}
\newcommand{\avec}{ \vec A}
\newcommand{\ap}{ \vec A_p}
\newcommand{\bb}{ \vec B}
\newcommand{\jj}{ \vec j}
\newcommand{\rr}{ \vec r}
\newcommand{\xx}{ \vec x}

% Definitions for the journal names
\newcommand{\adv}{    {\it Adv. Space Res.}} 
\newcommand{\annG}{   {\it Ann. Geophys.}} 
\newcommand{\aap}{    {\it Astron. Astrophys.}}
\newcommand{\aaps}{   {\it Astron. Astrophys. Suppl.}}
\newcommand{\aapr}{   {\it Astron. Astrophys. Rev.}}
\newcommand{\ag}{     {\it Ann. Geophys.}}
\newcommand{\aj}{     {\it Astron. J.}} 
\newcommand{\apj}{    {\it Astrophys. J.}}
\newcommand{\apjl}{   {\it Astrophys. J. Lett.}}
\newcommand{\apss}{   {\it Astrophys. Space Sci.}} 
\newcommand{\cjaa}{   {\it Chin. J. Astron. Astrophys.}} 
\newcommand{\gafd}{   {\it Geophys. Astrophys. Fluid Dyn.}}
\newcommand{\grl}{    {\it Geophys. Res. Lett.}}
\newcommand{\ijga}{   {\it Int. J. Geomagn. Aeron.}}
\newcommand{\jastp}{  {\it J. Atmos. Solar-Terr. Phys.}} 
\newcommand{\jgr}{    {\it J. Geophys. Res.}}
\newcommand{\mnras}{  {\it Mon. Not. Roy. Astron. Soc.}}
\newcommand{\nat}{    {\it Nature}}
\newcommand{\pasp}{   {\it Pub. Astron. Soc. Pac.}}
\newcommand{\pasj}{   {\it Pub. Astron. Soc. Japan}}
\newcommand{\pre}{    {\it Phys. Rev. E}}
\newcommand{\solphys}{{\it Solar Phys.}}
\newcommand{\sovast}{ {\it Soviet  Astron.}} 
\newcommand{\ssr}{    {\it Space Sci. Rev.}} 

%%%%%%%%%%%%%%%% Roadmap of the paper with all the labels %%%%%%%%%%%%%%%%%%%%%%%%
%%%% (with the exception of citations since author-year is easy to remember) %%%%%
1 Section~\ref{S-Introduction}
2 Section~\ref{S-general}
  Section~\ref{S-text}
  Section~\ref{S-labels}
3 Section~\ref{S-features}
  Section~\ref{S-equations}
   Section~\ref{S-simple-equations}
     Equation~(\ref{Eq-H-def})
   Section~\ref{S-array-equations}
     Equation~(\ref{Eq-A-B})
     Equation~(\ref{Eq-H-B})
     Equation~(\ref{Eq-H-N})
   Section~\ref{S-long-equations}
     Equation~(\ref{Eq-dH-Phi})  
  Section~\ref{S-figures}
  Section~\ref{S-tables}
  Section~\ref{S-references}
  Section~\ref{S-BibTeX}
  Section~\ref{S-Miscellaneous}
4 Section~\ref{S-Conclusion}

A Appendix~\ref{S-appendix}
 
 1 Figure~\ref{F-simple}
 2 Figure~\ref{F-4panels}
 3 Figure~\ref{F-rotate-cut}
 4 Figure~\ref{F-appendix}

 1 Table~\ref{T-simple}
 2 Table~\ref{T-complex}
 3 Table~\ref{T-appendix}

%%%%%%%%%%%%%%%%%%%% Results for the citation commands %%%%%%%%%%%%%%%%%%%%%%%%  
\cite{Kusano04}          : (Kusano \etal, 2004)
\cite{Kusano04,Berger03} : (Kusano \etal, 2004; Berger, 2003)
\inlinecite{Brown99}     : Brown, Canfield, and Pevtsov (1999)
\opencite{Brown99}       : Brown, Canfield, and Pevtsov, 1999
\citeauthor{Brown99}     : Brown, Canfield, and Pevtsov
\shortcite{Brown99}      : (1999)
\citeyear{Brown99}       : 1999
(\opencite{Berger84}, \citeyear{Berger03}) 
                         : (\opencite{Berger84}, \citeyear{Berger03})

